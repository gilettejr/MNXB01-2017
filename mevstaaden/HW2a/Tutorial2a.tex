\documentclass[a4paper,10pt,twoside]{article}

%\usepackage{ucs}
\usepackage[utf8]{inputenc}
%\usepackage{babel}
%\usepackage{fontenc}
\usepackage[pdftex]{graphicx}

\usepackage[pdftex]{hyperref}

\author{Mevine Staaden}
\title{Tutorial 2a Exercise Paper}
\date{09/14/17}

\begin{document}

  \maketitle

  \begin{center}
    \texttt{nat15mst@student.lu.se}
  \end{center}

\section{Introduction}
\label{sec:intro}

This is an introduction. The Summary will be given in Section \ref{sec:sum}

\section{About Linux}
\label{sec:Linux}

\begin{figure}[h]
  \begin{center}
   \includegraphics[width=6cm]{penguin.jpg}
   \caption{Penguin symbolises Linux. Source~\cite{penguin-image}}
   \label{fig:penguin}
  \end{center}
\end{figure}

Figure \ref{fig:penguin} shows a \textit{penguin}. For more details, check the Linux webpage~\cite{linux}.

\subsection{Linux flavours}
\label{sec:flavours}

Table~\ref{tab:flavours} lists some Linux flavours~\footnote{Only one is shown for simplicity}.

\begin{table}[h]
\begin{center}
 \begin{tabular}{c|c|c|c}
  \textbf{Distribution}&RedHat&Debian&SuSE\\ \hline \hline
  Fedora 20            & X    &      &    \\ \hline
 \end{tabular}
 \caption{Different flavours of Linux}
 \label{tab:flavours}
\end{center}
\end{table}

\section{About Mathematics}
\label{sec:math}

In-line math in \LaTeX \ is enclosed in \$ symbols. Backslash \textbackslash \ is used to denote special symbols.

Subscripts and Superscribts are always math: $A_x$, $A_{xy}$, $e^x$ and $e^{x^2}$. Using underscore \_ outside math without 
\textbackslash causes big\_ troubles. 

All special symbols are also math: $\alpha$, $\beta$, $\gamma$, $\delta$, $\sin x$, $\hbar$, $\lambda$, $\ldots$. More information
can be found in Ref.~\cite{latex}

\begin{equation}
\label{eq:chi2}
 \chi^2 = \sum\limits_i \left(\frac{F_i-D_i}{\sigma_i}\right)^2
\end{equation}



\section{Summary}
\label{sec:sum}

We learned the following:
\begin{itemize}
 \item Linux is good
 \item \LaTeX \ is good for:
  \begin{enumerate}
   \item Structuring Documents
   \item Writing mathematical equations
  \end{enumerate}
\end{itemize}

We can also write unformatted text using \texttt{verbatim} environment, but sometimes we have to specify this in the preamble:
\begin{verbatim}
 \usepackage{verbatim}
\end{verbatim}


\begin{thebibliography}{99}
  \bibitem{linux} Linux web site: \url{www.linux.com}
  \bibitem{latex} Leslie Lamport, \textsl{LaTeX: A Document Preparation System}, second edition, Addison-Wesley (1994)
  \bibitem{penguin-image} Penguin Image: \url{http://images.mentalfloss.com/sites/default/files/istock-511366776.jpg?resize=1100x740}
\end{thebibliography}


\end{document}
