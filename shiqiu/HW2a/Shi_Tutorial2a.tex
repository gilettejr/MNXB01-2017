\documentclass[a4paper,12pt,twoside]{article}

\usepackage{ucs}
\usepackage[utf8]{inputenc}
%\usepackage{babel}
\usepackage{fontenc}
\usepackage[pdftex]{graphicx}
\usepackage{amsmath}%I also use this maths package to demonstrate matrix
\usepackage[pdftex]{hyperref}

\author{Shi Qiu}
\title{Tutorial 2a exercise paper}
\date{09/14/17}

\begin{document}

\maketitle
\begin{center}
 \href{mailto:sh4722qi-s@student.lu.se}{\texttt{sh4722qi-s@student.lu.se}}
\end{center}
\tableofcontents
\newpage
\section{Introduction}\label{sec:intro}
This in introduction. Summary will be given in Section \ref{sec:sum}.

\section{About Linux}
\begin{figure}[ht]
 \begin{center}
  \includegraphics[scale=.15,trim=0cm 0cm 0 0]{penguin.png}
  \caption{This is a penguin!}
  \label{fig:penguin}
 \end{center}
\end{figure}


Figure \ref{fig:penguin} shows a \textit{king penguin}. For more details, check the Linux web page~\cite{linux}.

\subsection{Linux flavours}
Table~\ref{tab:flavours} lists some linux flavours~\footnote{Only one is shown for simplicity}.
\begin{table}[ht]\label{tab:flavours}
 \begin{center}
  \begin{tabular}{c||c|c|c}
   \hline
   \textbf{Distribution}&RedHat&Debian&SuSE\\ \hline\hline
   Fedora 20            & x    & y    & z  \\ \hline
  \end{tabular}
  \caption{Different flavours of Linux}
 \end{center}
\end{table}

\section{About mathematics}\label{sec:math}

In-line math in \LaTeX \ is enclosed in \$ symbols. Backslash \textbackslash \ is used to denote special symbols.

Subscripts and superscripts are always math: $A_x$, $A_{xy}$, $e^x$ and $e^{x^2}$. Using underscore \_ outside math without \textbackslash \ causes big\_troubles. 

All special symbols are also math: $\alpha$, $\beta$, $\gamma$, $\delta$, $\sin x$, $\hbar$, $\lambda$, $ldots$. More information can be found in Ref.~\cite{latex}.

Equation \ref{eq:chi2} shows $\chi^2$:
\begin{equation}\label{eq:chi2}
 \chi^2=\sum\limits_i \left(\frac{F_i-D_i}{\sigma_i}\right)^2
\end{equation}

We can also create a matrix
\begin{align}
 X = \begin{pmatrix}
     1 & x_1^{(1)} & x_2^{(1)}  & x_3^{(1)} \dots \\
     1 & x_1^{(2)} & x_2^{(2)}  & x_3^{(2)} \dots \\
     \vdots 
     \end{pmatrix}
\end{align}

To define a step function, we can do it like this
\begin{align}
 f(x)=\begin{cases}
       x+1  & y < 1
       \\
       -x+1 & y > 0
      \end{cases}
\end{align}

\section{Summary}\label{sec:sum}
We learned the following:
\begin{itemize}
 \item Linux is good
 \item \LaTeX \ is good for:
   \begin{enumerate}
    \item Structuring documents
    \item Writing mathematical equations
   \end{enumerate}
\end{itemize}

We can also write unformatted text using \texttt{verbatim} environment, but sometimes we have to specify this in the preamble:
\begin{verbatim}
 \usepackage{verbatim}
\end{verbatim}

\begin{thebibliography}{99}
 \bibitem{linux} Linux web site: \url{www.linux.com}
 \bibitem{latex} Leslie Lamport, \textsl{LaTeX: A Document Preparation System}, second edition, Addision-Wesley (1994).
\end{thebibliography}

\end{document}
