\documentclass[a4paper,12pt,twoside]{article}

\usepackage[pdftex]{graphicx}
\usepackage{caption,url,amsmath,bm}
\usepackage[retainorgcmds]{IEEEtrantools}
\usepackage[numbers,square,comma]{natbib}
\usepackage[pdftex]{hyperref}

\author{Viktor Drugge}
\title{Homework - Tutorial 2a}
\date{\today}

\begin{document}
\maketitle

\begin{center}
  \texttt{nat11vdr@student.lu.se}
\end{center}

\section{Introduction}
\label{sec:intro}
This is introduction. Summary should be given in Section \ref{sec:sum}.

\section{About Linux}
\label{sec:linux}
\begin{figure}[htb]
  \centering
  \includegraphics[width=2cm]{Tux.png}
  \caption{Tux the Linux logotype.}
  \label{fig:tux}
\end{figure}
Figure \ref{fig:tux} shows a \textit{penguine}. Fore more details, see the
Linux Web Page at \cite{linux}.

\subsection{Linux flavours}
\label{sec:flavorflave}
Table \ref{tab:flavorflave} lists some Linux flavours \footnote{Only one is
  shown for simplicity}.
\begin{table}[h]
  \centering
  \begin{tabular}{c|c|c|c}
    \textbf{Distribution} & RedHat & Debian & SuSE\\
    \hline\hline
    OpenSuSE 13.2 & & & X\\
    \hline
  \end{tabular}
  \caption{Different flavours of Linux}
  \label{tab:flavorflave}
\end{table}

\section{About mathematics}
\label{sec:math}
\LaTeX\ text and some math, more info can be found in reference \cite{latex}.
\begin{equation}
  \chi^2=\sum_i\left(\frac{F_i-D_i}{\sigma_i} \right)^2\label{eq:chi}
\end{equation}
Reference to equation \eqref{eq:chi}.
\begin{verbatim}
\usepackage{verbatim}
Environment where latex does not interpret content.
\end{verbatim}

\section{Summary}
\label{sec:sum}
We learned the following:
\begin{itemize}
  \item Linux is good
  \item \LaTeX\ is good for:
    \begin{enumerate}
      \item Structuring documents
      \item Writing mathematical equations
    \end{enumerate}
\end{itemize}

\section{References} %compile u/ pdflatex-bibtex-pdflatex-pdflatex
\nocite{*}
\begingroup
\renewcommand{\section}[2]{} %Remove default ``Reference'' title
\bibliography{Refs.bib}
\endgroup
\bibliographystyle{unsrtnat}
\end{document}
