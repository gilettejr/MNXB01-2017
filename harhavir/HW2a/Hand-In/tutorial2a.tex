\documentclass[a4paper,10pt,twoside]{article}

\usepackage{ucs}
\usepackage[utf8]{inputenc}
\usepackage[pdftex]{graphicx}

\usepackage[pdftex]{hyperref}

\author{Harald Havir}
\title{Tutorial 2a exercise paper}
\date{2017-09-14}

\begin{document}
 \maketitle
 
 \begin{center}
  \texttt{fte15hh1@student.lu.se}
 \end{center}
 \section{Introduction}
 \label{sec:intro}
 This is an introduction. summary will be given in Section \ref{sec:sum}
 
 \section{About Linux}
 \label{sec:linux}
 
 \begin{figure}[h]
 \begin{center}
  \includegraphics[width=2cm]{Tux.png}
  \caption{Penguin symbolises Linux}
  \label{fig:penguin}
 \end{center}
  
 \end{figure}

 Figure \ref{fig:penguin} shows a \textit{penguin}. For more details, check the Linux Web page~\cite{linux}
 \subsection{Linux Flavours}
  \label{sec:flavours}
  
  Table~\ref{tab:flavours} lists some Linux flavours~\footnote{Only one is shown for simplicity}
  
  \begin{table}[h]
   \begin{center}
    
    \label{tab:flavours}
    \begin{tabular}{c|c|c|c} % | is for border, c is for centering.
     \textbf{Distribution} & RedHat & Debian & SuSE \\ \hline \hline
      Fedora 20 	& X	& 	&	\\ \hline
    \end{tabular}
    \caption{Different flavours of Linux}
   \end{center} 
  \end{table}

 \newpage


  \section{About mathematics}
  \label{sec:math}
  
  in-line math in \LaTeX \ is enclosed in \$ symbols. Backslash \textbackslash \ is used to denote special symbols. 
  
  Subscrits and superscripts are always math: $A_x$, $A_{xy}$, $e^x$ and $e^{x^2}$. Using underscore \_ outside math without \textbackslash causes big\_troubles.
  
  All special symbols are also math: $\alpha$, $\beta$, $\gamma$, $\delta$, $\sin x$, $\hbar$, $\lambda$, $\ldots$ More information can be found in Ref.~\cite{latex}.
  
  Equation~\ref{eq:chi2} shows $\chi^2$
  
  \begin{equation}
   \label{eq:chi2}
    \chi^2 = \sum\limits_i \left(\frac{F_i-D_i}{\sigma_i}\right)^2
  \end{equation}

 \section{Summary}
  \label{sec:sum}
   We learned the following:
   \begin{itemize}
    \item Linux is good
    \item \LaTeX \ is good for:
    \begin{enumerate}
     \item Structuring documents
     \item Writing mathematical equations
    \end{enumerate}

   \end{itemize}

We can also write unformatted text using \texttt{verbatim} environment, but sometimes we have to specify this in the preamble:
\begin{verbatim}
 \usepackage{verbatim}
\end{verbatim}

   
   
 \begin{thebibliography}{99}
  \bibitem{linux} Linux web site: \url{www.linux.com}
  \bibitem{latex} Leslie Lamport, \textsl{LaTeX: A Document Preparation System}, second edition, Addison-Wesley (1994).
 \end{thebibliography}
\end{document}
