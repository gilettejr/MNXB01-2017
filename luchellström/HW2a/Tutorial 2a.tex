\documentclass[a4paper]{article}
\usepackage[T1]{fontenc}
\usepackage[utf8]{inputenc}
\usepackage[swedish,english]{babel}
\usepackage{polynom}
\usepackage{booktabs}
\usepackage{amsmath}
\usepackage{amssymb}
\usepackage{hyperref}
\usepackage{mathtools}
\usepackage{graphicx}
\usepackage{color}
\usepackage{subfig}
\usepackage{csquotes}
\usepackage{eso-pic}
\usepackage{siunitx}

\title{Tutorial 2a exercise paper}
\author{Lucas Hellström}
\date{13/9 - 2017}

\AddToShipoutPicture{%
  \AtPageUpperLeft{%
    \hspace*{20pt}\makebox(200,-20)[lt]{%
      \footnotesize%
      \textbf{Lucas Hellström - 950905-0655}%
}}}

\begin{document}

\maketitle

\begin{center}
	\texttt{nat14lhe@student.lu.se}
\end{center}

\section{Introduction}
\label{sec:intro}

This is the introduction. Summary will be given in Section \ref{sec:sum}

\section{About Linux}

	\begin{figure}[h]
		\centering
		\includegraphics[width=0.25\textwidth]{Tux.png}
		\caption{Penguin symbolises Linux}
		\label{fig:penguin}
	\end{figure}

	Figure \ref{fig:penguin} shows a \textit{penguin}. For more details, check the Linux Web page ~\cite{linux}
	
	\subsection{Linux flavours}
	\label{sec:flavours}
		Table ~\ref{tab:flavours} lists some Linux flavours~\footnote{Only one is shown for simplicity}.
	\begin{table}[h]
		\begin{center}
			\begin{tabular}{cccc} \toprule[1.5pt]
				\textbf{Distribution}&RedHat&Debian&SuSe\\ \midrule
				Fedora 23 & X & & \\ \bottomrule[1.5pt]
			\end{tabular}
			\caption{Different flavours of Linux}
			\label{tab:flavours}
		\end{center}
	\end{table}

\section{About mathematics}
\label{sec:math}

	In-line math in \LaTeX \ is enclosed in \$ symbols. Backslash \textbackslash \ is used to denote special symbols.

	Subscripts and superscripts are always math: $A_x$, $A_{xy}$, $e^x$ and $e^{x^2}$. Using underscore \_ outside math without \textbackslash \ causes big\_trouble.

	All special symbols are also math: $\alpha,\; \beta,\; \gamma,\; \delta,\; \sin x,\; \hbar,\; \lambda,\; \ldots$ More information can be found in Ref.~\cite{latex}

	Equation ~\ref{eq:chi2} shows $\chi^2$.

	\begin{equation}
	\label{eq:chi2}
		\chi^2 = \sum_i \left(\frac{F_i-D_i}{\sigma_i}\right)^2
	\end{equation}

\section{Summary}
\label{sec:sum}

	We learned the following: 

	\begin{itemize}
		\item Linux is good
		\item \LaTeX \ is good
		\begin{enumerate}
			\item Structuring documents
			\item Writing mathematical equations
		\end{enumerate}
	\end{itemize}

	We can also write unformatted text using the \texttt{verbatim} environment, but sometimes we have to specify this in the preamble:
	
	\begin{verbatim}
	\usepackage{verbatim}
	\end{verbatim}

\begin{thebibliography}{99}
	\bibitem{linux} Linux web site: \url{www.linux.com}
	\bibitem{latex} Leslie Lamport, \textit{LaTeX: A Document Preparation System}, second edition, Addison-Wesley (1994)
\end{thebibliography}

\end{document}