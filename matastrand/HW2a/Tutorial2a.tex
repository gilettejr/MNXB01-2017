\documentclass[a4paper,10pt,twoside]{article}

%\usepackage{ucs}
\usepackage[utf8]{inputenc}
%\usepackage[UKenglish]{babel}
\usepackage[pdftex]{graphicx}

\usepackage[pdftex]{hyperref}

\author{Mattias Åstrand}
\title{Tutorial 2a exercise paper}
\date{09/14/17}

\begin{document}

\maketitle

\begin{center}
 \texttt{mattias.astrand98@gmail.com}
\end{center}

\section{Introduction}
\label{sec:intro}
This is an intro. A summary will be given in section \ref{sec:sum}.

\section{About Linux}
\label{sec:linux}

\begin{figure}[h]
\begin{center}
  \includegraphics[width=2cm]{Tux.png}
  \caption{Tux is the mascotte of Linux.}
  \label{fig:penguin}
\end{center}
\end{figure}

Figure \ref{fig:penguin} shows a \textit{penguin}. For more details, check the Linux Web page~\cite{linux}.

\subsection{Linux Flavours}
\label{sec:Flavours}

Table~\ref{tab:flavours} lists some Linux flavours~\footnote{Only one is shown for simplicity.}.

\begin{table}[h]
\begin{center}
 \begin{tabular}{c|c|c|c}
 
  \hline \hline \textbf{Distribution}&RedHat&Debian&SuSE\\ \hline
	Fedora 20	&  X   &     &    \\ \hline \hline
	
 \end{tabular}
 \caption{Different flavours of Linux}
 \label{tab:flavours}
\end{center}
\end{table}

\section{About Mathematics}
\label{sec:math}

In-line math in \LaTeX \ is enclosed in \ symbols. Backslash \textbackslash \ is used to denote special symbols.

Subscripts and superscripts are always math: $A_x$, $A_{xy}$,
$e^x$ and $e^{x^2}$. Using underscore \_ \ outside math without \textbackslash \ causes big\_troubles.

All special symbols are also math: $\alpha$, $\beta$, $\gamma$, $\delta$, $\sin x$, $\hbar$, $\lambda$, $\ldots$ More information can be found in Ref.~\cite{latex}.

Equation~\ref{eq:chi2} shows $\chi^2$.

\begin{equation}
 \label{eq:chi2}
 \chi^2 = \sum\limits_i \left(\frac{F_i - D_i}{\sigma_i}\right)^2
\end{equation}


\section{Summary}
\label{sec:sum}

We learned the following:
\begin{itemize}
 \item Linux is your friend.
 \item \LaTeX \ is good for:
	\begin{enumerate}
	 \item Structuring documents.
	 \item Writing mathematical equations.
	\end{enumerate}
\end{itemize}

We can also write unformatted text using \texttt{verbatim} environment, but sometimes we have to specify this in the preamble:
\begin{verbatim}
 \usepackage{verbatim}
\end{verbatim}

 
\begin{thebibliography}{99}
 \bibitem{linux} Linux web site: \url{www.linux.com}
 \bibitem{latex} Leslie Lamport, \textsl{LaTeX: A Document Preparation System}, second edition, Addison-Wesley (1994). 
\end{thebibliography}

 
\end{document}
